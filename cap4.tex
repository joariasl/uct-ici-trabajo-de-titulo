\chapter{Diseño y Especificaciones de la Solución}

\section{Arquitectura y Componentes del Sistema}

\subsection{Componentes}

La solución está compuesta por dos bibliotecas Java principales: \texttt{sii-dte-lib-java} y \texttt{sii-schemas-lib-java}.

El Cuadro~\ref{tab:componentesSolucion} presenta una descripción detallada de cada uno de los componentes.

\begin{table}
    \centering
    \caption{Componentes de la solución}
    \begin{tabularx}{\linewidth}{|p{0.3\linewidth}|X|}
        \hline
        \textbf{Componente} &
        \textbf{Descripción}\\
        \hline
        sii-schemas-lib-java & Biblioteca con modelos Java generados desde shemas de Formato XML de Documentos Electrónicos distribuidos por el SII Chile \\
        \hline
        sii-dte-lib-java & Biblioteca Java cliente para el manejo y envío de Documentos Electrónicos del SII Chile \\
        \hline
        Proyecto Java & Proyecto Java encargado de realizar procesos de negocio para el manejo de los DTE de una o varias empresas \\
        \hline
    \end{tabularx}
    \label{tab:componentesSolucion}
\end{table}

\subsection{Modelo de dependencias para Biblioteca}

La Figura~\ref{figure:dependenciasComponentes} detalla la relación de dependencias entre los componentes de la solución. \texttt{Proyecto Java} representa cualquier implementación de las bibliotecas, ya sea como un cliente, servicio API o microservicio.

\begin{figure}[H]
    \centering
    \begin{tikzpicture}
        \graph [nodes={draw}] {
            Proyecto Java,
            Proyecto Java -> "sii-dte-lib-java",
            "sii-dte-lib-java" -> "sii-schemas-lib-java";
        };
    \end{tikzpicture}
    \caption{Dependencia entre componentes.}
    \label{figure:dependenciasComponentes}
\end{figure}

\section{Patrones de Diseño}

\section{Herramientas utilizadas}

\subsection{Recursos para Java}

El Cuadro~\ref{tab:recursosJava} detalla los recursos utilizados para el desarrollo de las bibliotecas Java que dan soporte a los esquemas proporcionados por el SII.

\begin{table}
    \centering
    \caption{Recursos para Java}
    \begin{tabularx}{\linewidth}{|p{0.3\linewidth}|X|}
        \hline
        \textbf{Recurso} &
        \textbf{Descripción}\\
        \hline
        JAXB & Jakarta XML Binding ofrece a los desarrolladores de Java una forma eficiente y estándar de mapear XML a código Java. \textcite{jaxbri} \\
        \hline
        Jakarta EE & Framework Open Source para el desarrollo de Cloud Native Java Applications. Define un conjunto de especificaciones para el desarrollo de aplicaciones empresariales. Es la evolución de Java EE (Java Platform, Enterprise Edition) después de que Oracle transfiriera el proyecto a la Eclipse Foundation en Septiembre 12 de 2017. \textcite{jakartaee} \\
        \hline
        Jakarta XML Binding & Jakarta XML Binding define una API y herramientas que automatizan el mapeo entre documentos XML y objetos Java. \\
        \hline
        Jakarta Annotations & Jakarta Annotations define una colección de anotaciones que representan conceptos semánticos comunes que permiten un estilo declarativo de programación que se aplica a una variedad de tecnologías Java. \\
        \hline
    \end{tabularx}
    \label{tab:recursosJava}
\end{table}

\section{Licenciamiento}
