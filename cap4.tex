\chapter{Diseño y Especificaciones de la Solución}

\section{Arquitectura y Componentes del Sistema}

\subsection{Componentes}

La solución está compuesta por dos bibliotecas Java principales: \texttt{sii-dte-lib-java} y \texttt{sii-schemas-lib-java}.

El Cuadro \ref{tab:componentesSolucion} presenta una descripción detallada de cada uno de los componentes.

\begin{table}
    \centering
    \caption{Componentes de la solución}
    \begin{tabularx}{\linewidth}{|p{0.3\linewidth}|X|}
        \hline
        \textbf{Componente} &
        \textbf{Descripción}\\
        \hline
        sii-schemas-lib-java & Biblioteca con modelos Java generados desde shemas de Formato XML de Documentos Electrónicos distribuidos por el SII Chile \\
        \hline
        sii-dte-lib-java & Biblioteca Java cliente para el manejo y envío de Documentos Electrónicos del SII Chile \\
        \hline
        Proyecto Java & Proyecto Java encargado de realizar procesos de negocio para el manejo de los DTE de una o varias empresas \\
        \hline
    \end{tabularx}
    \label{tab:componentesSolucion}
\end{table}

\subsection{Modelo de dependencias para Biblioteca}

La Figura \ref{figure:dependenciasComponentes} detalla la relación de dependencias entre los componentes de la solución. \texttt{Proyecto Java} representa cualquier implementación de las bibliotecas, ya sea como un cliente, servicio API o microservicio.

\begin{figure}[H]
    \centering
    \begin{tikzpicture}
        \graph [nodes={draw}] {
            Proyecto Java,
            Proyecto Java -> "sii-dte-lib-java",
            "sii-dte-lib-java" -> "sii-schemas-lib-java";
        };
    \end{tikzpicture}
    \caption{Dependencia entre componentes.}
    \label{figure:dependenciasComponentes}
\end{figure}

\section{Patrones de Diseño}

\section{Herramientas utilizadas}

\section{Licenciamiento}
