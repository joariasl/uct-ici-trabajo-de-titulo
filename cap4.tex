\chapter{Discusión y Análisis de Resultados}

Los resultados son concluyentes, los efectos directos sugieren que la subvención al transporte (\texttt{I14}) no tiene ningún efecto sobre reducir la contaminación del aire en áreas residenciales. Por otro lado, la educación (\texttt{I11}), I+D investigación y desarrollo (\texttt{I13}) y la mitigación (\texttt{I16}), si tienen un efecto directo sobre reducir la contaminación del aire. Los efectos olvidados sugieren que si la gente mejora el aislamiento térmico de sus casas (\texttt{B1}), puede contribuir a una reducción mayor de la contaminación del aire. Los sistemas de evaluación de impacto ambiental (\texttt{I15}), educación (\texttt{I11}) y I+D investigación y desarrollo (\texttt{I13}) también pueden desempeñar un papel importante para fomentar el cambio de comportamiento ya que aparecen como intermediarios entre las relaciones de origen asociadas a los incentivos, mejora y producción (\texttt{I8}), subvención para leña (\texttt{I9}) y el destino asociado al comportamiento \texttt{B1}