\chapter{Introducción}
\section{Problema}
%% cite{Infórmate} como \texcite o como \parencite
Problema. Debe reflejar lo que lleva a hacer el trabajo de título.

\section{Estado del Arte}
Estado del arte. Debe mostrar tanto los trabajos que dieron inicio a la línea de investigación (aunque deben ser pocos) como los trabajos más recientes sobre los que se quiere avanzar o al menos apoyar. Deben citarse usando el estilo de cita seleccionado. El estilo de cita debe ser homogéneo a los largo del trabajo de título. Cada uno de los trabajo citados debe ser acompañado de una breve explicación sobre sus objetivos, métodología y conclusiones. Dependiendo del estilo de escritura requerido por el profesor guía, se deberá explicar cuá es la mejora del trabajo de título sobre cada uno de los trabajos citados en el estado del arte.

Por ejemplo: Desde sus inicios, la teoría de los efectos olvidados ha sido aplicada en problemas de gestión. Como por ejemplo, en el ámbito político \parencite[105-126]{kaufmann1988modelos} o incluso el ámbito financiero \parencite[127-151]{kaufmann1988modelos}. 

Actualmente diversos autores han usado la teoría de los efectos olvidados en áreas como por ejemplo, la educación. \textcite{cabrera2020analysis} realizaron un análisis del aprendizaje basado en problemas de rendimiento académico. Se evidenció que los factores de causa-efecto como la gestión muy teórica, o sus contenidos, no contribuyen a mejorar el perfil de graduación de los estudiantes y que tampoco ofrecen soluciones reales a la sociedad. No obstante, este trabajo no ofrece escenarios más allá del evaluado por los informantes clave, aspecto que incorporamos en nuestro desarrollo informático.  \\

\section{Objetivos}
\subsection{Objetivo general}
\begin{itemize} 
    \item Objetivo general. Comenzar por un solo verbo en infinitivo. Debe pertenecer a los dos niveles de la taxonomía de Bloom, modificada \textcite{anderson2001taxonomy} \footnote{Ver \url{https://www3.gobiernodecanarias.org/medusa/edublog/cprofestenerifesur/wp-content/uploads/sites/105/2015/12/Captura-de-pantalla-2015-12-03-a-las-22-12-56.png}}. Suele usarse como título del proyecto sin el verbo.
\end{itemize}
\subsection{Objetivos específicos}
\begin{itemize}
    \item Objetivo específico 1. Comenzar por un solo verbo en infinitivo. Debe pertenecer al mismo nivel o a uno más bajo de la taxonomía de Bloom, modificada \textcite{anderson2001taxonomy}. No confundir con una tarea, pues el objetivo específico está compuesto de múltiples tareas.
    \item Objetivo específico 2. Comenzar por un solo verbo en infinitivo. Debe pertenecer al mismo nivel o a uno más bajo de la taxonomía de Bloom, modificada \textcite{anderson2001taxonomy}. No confundir con una tarea, pues el objetivo específico está compuesto de múltiples tareas.
    \item Objetivo específico 3. Comenzar por un solo verbo en infinitivo. Debe pertenecer al mismo nivel o a uno más bajo de la taxonomía de Bloom, modificada \textcite{anderson2001taxonomy}. No confundir con una tarea, pues el objetivo específico está compuesto de múltiples tareas.
    \item Etc. Tratar de no tener más de tres objetivos específicos pues serán estos de los que deberán dar cuenta en las conclusiones.
\end{itemize}