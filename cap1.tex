\chapter{Introducción}
\section{Problema}
%% cite{Infórmate} como \texcite o como \parencite
La Ley Nº 20.727 de 2014 establece el uso obligatorio de la factura electrónica, junto a otros documentos tributarios electrónicos como liquidación factura, notas de débito y crédito y factura de compra. Para las grandes empresas el plazo de incorporación venció en 2014, para las Medianas y Pequeñas Empresas el año 2016 y 2017 y para las Microempresas hasta el año 2017 y 2018.

Ésta implementación posiciona a Chile como uno de los países con mayor y más temprana adopción a la digitalización en Latinoamérica, lo que impulsa a las empresas a tener mejores procesos de gestión de la información tanto interna como a la de dar cumplimiento a las regulaciones fiscales del país. Sin embargo, también conllevan a establecer una barrera de entrada para nuevas empresas, las que deben ser digitales desde el día cero para comenzar a operar. Y es un desafío constante a mantener un sistema que permita dar cumplimiento con lo que dispone el Servicio de Impuestos Internos de Chile en cuanto a dichas regulaciones de manejo y presentación de documentos tributarios.

\section{Estado del Arte}
Estado del arte. Debe mostrar tanto los trabajos que dieron inicio a la línea de investigación (aunque deben ser pocos) como los trabajos más recientes sobre los que se quiere avanzar o al menos apoyar. Deben citarse usando el estilo de cita seleccionado. El estilo de cita debe ser homogéneo a los largo del trabajo de título. Cada uno de los trabajo citados debe ser acompañado de una breve explicación sobre sus objetivos, métodología y conclusiones. Dependiendo del estilo de escritura requerido por el profesor guía, se deberá explicar cuá es la mejora del trabajo de título sobre cada uno de los trabajos citados en el estado del arte.

Por ejemplo: Desde sus inicios, la teoría de los efectos olvidados ha sido aplicada en problemas de gestión. Como por ejemplo, en el ámbito político \parencite[105-126]{kaufmann1988modelos} o incluso el ámbito financiero \parencite[127-151]{kaufmann1988modelos}. 

Actualmente diversos autores han usado la teoría de los efectos olvidados en áreas como por ejemplo, la educación. \textcite{cabrera2020analysis} realizaron un análisis del aprendizaje basado en problemas de rendimiento académico. Se evidenció que los factores de causa-efecto como la gestión muy teórica, o sus contenidos, no contribuyen a mejorar el perfil de graduación de los estudiantes y que tampoco ofrecen soluciones reales a la sociedad. No obstante, este trabajo no ofrece escenarios más allá del evaluado por los informantes clave, aspecto que incorporamos en nuestro desarrollo informático.  \\

\section{Objetivos}
\subsection{Objetivo general}
\begin{itemize} 
    \item Desarrollar un recurso para la implementación de Documentación Tributaria Electrónica que cumpla con los estándares exigidos por el Servicio de Impuestos Internos y permita a las empresas dar cumplimiento a su obligación de adopción según las normas y requisitos establecidos, además de gestionar y resguardar eficazmente dicha documentación tributaria e información asociada, escalable al uso e implementación tanto en grandes empresas como en PyME y Microempresas.
\end{itemize}
\subsection{Objetivos específicos}
\begin{itemize}
	\item Analizar los sistemas y recursos informáticos actuales que emplean las empresas para gestionar documentos fiscales y determinar cómo los implementan.
	\item Estudiar las necesidades y estado actual de las empresas en cuanto a la normativa vigente para cumplir con la documentación electrónica.
	\item Desarrollar una librería y/o servicio informático que permita realizar las operaciones que indica el Servicio de Impuestos Internos para la implementación de Documentación Tributaria Electrónica.
	\item Implementar estándares técnicos que garanticen calidad, seguridad, escalabilidad y resiliencia.
	\item Validar que la solución desarrollada pueda ser integrada técnicamente en los sistemas de las empresas.
\end{itemize}