\chapter{Introducción}

\section{Problema}

%% cite{Infórmate} como \texcite o como \parencite
La Ley Nº 20.727 de 2014 establece el uso obligatorio de la factura electrónica, junto a otros documentos tributarios electrónicos como liquidación factura, notas de débito y crédito y factura de compra. Para las grandes empresas el plazo de incorporación venció en 2014, para las Medianas y Pequeñas Empresas el año 2016 y 2017 y para las Microempresas hasta el año 2017 y 2018.

Esta implementación posiciona a Chile como uno de los países con mayor y más temprana adopción a la digitalización en Latinoamérica, lo que impulsa a las empresas a mejorar los procesos de gestión de la información, tanto internamente como en el cumplimiento de las regulaciones fiscales del país. Sin embargo, también conllevan a establecer una barrera de entrada para nuevas empresas, las que deben ser digitales desde el día cero para comenzar a operar. Y es un desafío constante a mantener un sistema que permita dar cumplimiento con lo que dispone el Servicio de Impuestos Internos de Chile en cuanto a dichas regulaciones de manejo y presentación de documentos tributarios.

De acuerdo con las estadísticas publicadas por el Servicio de Impuestos Internos (SII) en su portal oficial sobre Factura Electrónica \textcite{estadisticasFacturaElectronica} \footnote{Ver \url{https://www.sii.cl/servicios_online/1039-estadistic-1182.html}}, se destacan los siguientes datos relevantes:
\begin{enumerate}
	\item Durante el año 2023, se inscribieron en Factura Electrónica 137.714 empresas.
	\item De los contribuyentes inscritos durante el año 2023, el 94,56\% lo hizo a través del Sistema de Facturación Gratuito del SII.
	\item Durante el año 2023, se emitieron más de 695 millones de Documentos Tributarios Electrónicos.
	\item El 9\% de las empresas habilitadas utilizan software de mercado o propio, mientras que el otro 91\% utiliza el Sistema de Facturación gratuito otorgado por el SII.
	\item El 79\% de las empresas habilitadas corresponde a grandes empresas, mientras que el 21\% corresponde a Medianas y Pequeñas Empresas, y Microempresas. Esto sugiere que la mayoría de las grandes empresas aún no implementan una solución de software propio que se integre directamente con sus sistemas de gestión internos.
	\item En el año 2024, únicamente 11.076 empresas implementaron un software de mercado o propio, mientras que 85.583 fueron habilitados como emisores de factura electrónica a través del Sistema de Facturación Gratuito del SII.
	\item El 58\% de los DTE emitidos corresponden únicamente al de tipo Factura Electrónica.
\end{enumerate}

La obligación de cumplir con la emisión de DTE, junto con la evidencia de la baja adopción de integraciones con software de mercado o propio —un hecho que se contradice con las necesidades de un comercio interconectado y beneficiado por la digitalización—, hace necesario desarrollar una solución que permita a las empresas implementar y gestionar eficientemente sus DTE, además de integrarlas con la amplia variedad de soluciones digitales disponibles para la gestión de procesos y comercio.

Los constantes avances tecnológicos continúan siendo una parte crucial en la optimización de los costos operativos para las empresas. Estas han ido evolucionando, haciendo cada vez más necesaria una mayor integración entre sistemas de la misma o diferentes entidades. Actualmente, la incorporación de análisis de datos avanzados y la inteligencia artificial se han ido adoptando en búsqueda de optimizar la toma de decisiones, aumentar la precisión y eficiencia, y avanzar hacia la automatización y autonomía en la gestión y operación de las empresas.

\section{Antecedentes}

En el mercado de software empresarial actual, existe una gran cantidad de programas que permiten mejorar la gestión empresarial tanto en la parte administrativa como en el proceso. Existen sistemas gratuitos como Odoo (ex OpenERP) que permiten gestionar la información de una empresa, están calificados como sistemas  de clase mundial, pero que por lo mismo no están concretamente diseñados para adaptarse a la normativa comercial y tributaria chilena y no adoptan de manera nativa la actual integración de documentación tributaria electrónica que exige el Servicio de Impuestos Internos.

También existen sistemas de pago ofrecidos por empresas como SAP, Oracle, QAD, Softland, Microsoft, entre otros; pero que no logran llegar a las MiPyME (Pequeñas Medianas y Microempresas) debido a su alto costo de mercado, plataforma y mantención. Y para algunas de las grandes empresas, estas no permiten realizar una integración adecuada a sistemas informáticos propios o no les permite adoptar rápidamente los cambios necesarios cuando el Servicio de Impuestos Internos aplica nuevos requisitos a la regulación.

Para facilitar la adopción a las PyME y Microempresas el Servicio de Impuestos Internos ha puesto a disposición de los contribuyentes un sistema de facturación gratuito de funcionalidad básica, que les permite operar con facturas electrónicas y cumplir con la normativa que el SII ha establecido para los contribuyentes autorizados a emitir documentos tributarios electrónicos, pero que no permite integrar funcionalidades acordes a las necesidades de cada empresa y compatibles con sus propios sistemas.

Como solución para la implementación en el desarrollo de software propio, existió una librería DTE para Java creada por NicLabs en el año 2011. Actualmente se encuentra deprecada desde Enero del 2017 y ya no recibe actualizaciones para la solución de incidencias, pero aparentemente funcional aún corrigiendo algunas definiciones deprecadas u otras incidencias, pero sin ninguna garantía.

Empresas como Acepta.com S.A. y SOVOS ofrecen soluciones de facturación electrónica de pago en forma de Web Service, enfocada principalmente en el segmento de mercado de las medianas y grandes empresas. Aunque, según la evidencia de adopción de software de mercado, el Software de Facturación gratuito ofrecido por el SII continúa siendo la solución predominante para complir con la obligación, a pesar de sus limitaciones.

\section{Objetivos}

\subsection{Objetivo general}

\begin{itemize} 
    \item Desarrollar un recurso para la implementación de Documentación Tributaria Electrónica que cumpla con los estándares exigidos por el Servicio de Impuestos Internos y permita a las empresas dar cumplimiento a su obligación de adopción según las normas y requisitos establecidos, además de gestionar y resguardar eficazmente dicha documentación tributaria e información asociada, escalable al uso e implementación tanto en grandes empresas como en PyME y Microempresas.
\end{itemize}

\subsection{Objetivos específicos}

\begin{itemize}
	\item Analizar los sistemas y recursos informáticos actuales que emplean las empresas para gestionar documentos fiscales y determinar cómo los implementan.
	\item Estudiar las necesidades y estado actual de las empresas en cuanto a la normativa vigente para cumplir con la documentación electrónica.
	\item Desarrollar una librería y/o servicio informático que permita realizar las operaciones que indica el Servicio de Impuestos Internos para la implementación de Documentación Tributaria Electrónica.
	\item Implementar estándares técnicos que garanticen calidad, seguridad, escalabilidad y resiliencia.
	\item Validar que la solución desarrollada pueda ser integrada técnicamente en los sistemas de las empresas.
\end{itemize}